\documentclass[10pt,twocolumn,letterpaper]{article}

\usepackage{cvpr}
\usepackage{times}
\usepackage{epsfig}
\usepackage{graphicx}
\usepackage{amsmath}
\usepackage{amssymb}

% Include other packages here, before hyperref.

% If you comment hyperref and then uncomment it, you should delete
% egpaper.aux before re-running latex.  (Or just hit 'q' on the first latex
% run, let it finish, and you should be clear).
\usepackage[pagebackref=true,breaklinks=true,letterpaper=true,colorlinks,bookmarks=false]{hyperref}

% \cvprfinalcopy % *** Uncomment this line for the final submission

\def\cvprPaperID{****} % *** Enter the CVPR Paper ID here
\def\httilde{\mbox{\tt\raisebox{-.5ex}{\symbol{126}}}}

% Pages are numbered in submission mode, and unnumbered in camera-ready
\ifcvprfinal\pagestyle{empty}\fi
\begin{document}

%%%%%%%%% TITLE
\title{\LaTeX\ Author Guidelines for CVPR Proceedings}

\author{Raphael Michel\\
Institution1\\
Institution1 address\\
{\tt\small firstauthor@i1.org}
% For a paper whose authors are all at the same institution,
% omit the following lines up until the closing ``}''.
% Additional authors and addresses can be added with ``\and'',
% just like the second author.
% To save space, use either the email address or home page, not both
\and
Lucas-Raphael Müller\\
Institution2\\
First line of institution2 address\\
{\tt\small secondauthor@i2.org}
\and
FLorian Störtz\\
Institution2\\
First line of institution2 address\\
{\tt\small secondauthor@i2.org}
}

\maketitle
%\thispagestyle{empty}

%%%%%%%%% ABSTRACT
\begin{abstract}
   Tbd.
\end{abstract}

%%%%%%%%% BODY TEXT
\section{Introduction}

%-------------------------------------------------------------------------
\section{Methods}

\subsection{Vision}
We provide a framework which features robust and fast extraction of game information(i.e. board configuration).
The board is identified on the photograph by color and pattern comparision.
The image is then cropped and undergoes various image processing steps, such as wrapping, contrast enhancement and color channel extraction.
The current board configuration can then be read and converted to a computer readable format by thresholded color comparision.




%-------------------------------------------------------------------------
\section{Results}

\subsection{Vision}
Processing time is $< 0.3$ s (i7-4790K CPU @ 4.00GHz) per image.


%-------------------------------------------------------------------------
\section{Discussion}

{\small
\bibliographystyle{ieee}
\bibliography{egbib}
}

\end{document}
